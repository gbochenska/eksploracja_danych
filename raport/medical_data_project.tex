\documentclass[onecolumn,12pt]{article}

%----Wgranie packages----------
\setlength{\voffset}{-0.55in}
\setlength{\headsep}{3pt}

\usepackage{hyperref}
\hypersetup{
    colorlinks=false, %set true if you want colored link
    linktoc=all,     %set to all if you want both sections and subsections linked
}

\usepackage[polish]{babel}
\usepackage{algorithm,algorithmic,lipsum}
\usepackage[utf8]{inputenc}
\usepackage{url}
\usepackage{anyfontsize}
\usepackage{multirow}
\usepackage{subfigure}
\usepackage{tabularx}
\usepackage{ragged2e}
\usepackage{booktabs}
\usepackage{multirow}
\usepackage{grffile}
\usepackage{indentfirst}
\usepackage{caption}
\usepackage{listings}
\usepackage{lipsum}
\usepackage{enumitem}
%\usepackage{xcolor}
%\usepackage{hyperref}
\usepackage{catchfilebetweentags}
\usepackage[smartEllipses]{markdown}
\usepackage[ruled,linesnumbered,lined]{algorithm2e}
\usepackage[bookmarks=false]{hyperref}
\usepackage{mathtools}
\DeclarePairedDelimiter\ceil{\lceil}{\rceil}
\DeclarePairedDelimiter\floor{\lfloor}{\rfloor}

\hypersetup{colorlinks,
    linkcolor=blue,
    citecolor=blue,
    urlcolor=blue}

\usepackage[svgnames]{xcolor}
\usepackage{inconsolata}
\usepackage{fontawesome}

\usepackage{csquotes}
\DeclareQuoteStyle[quotes]{polish}
    {\quotedblbase}
    {\textquotedblright}
    [0.05em]
    {\quotesinglbase}
    {\fixligatures\textquoteright}
\DeclareQuoteAlias[quotes]{polish}{polish}

\usepackage[nottoc]{tocbibind}
\usepackage[
style=numeric,
sorting=nyt,
isbn=false,
doi=true,
url=true,
backref=false,
backrefstyle=none,
maxnames=10,
giveninits=true,
abbreviate=true,
defernumbers=false,
backend=biber]{biblatex}

\lstset{
        %language=Python,  %%  PHP,  C,  Java,  etc.
        basicstyle=\ttfamily\footnotesize,
        backgroundcolor=\color{gray!5},
        commentstyle=\it\color{Green},
        keywordstyle=\color{Red},
        stringstyle=\color{Blue},
        numberstyle=\tiny\color{Black},        
        %  morekeywords={TestKeyword},
        %  mathescape=true,
        escapeinside=`',
        frame=single,  %shadowbox,  
        tabsize=2,
        rulecolor=\color{black!30},
        title=\lstname,
        breaklines=true,
        breakatwhitespace=true,
        framextopmargin=2pt,
        framexbottommargin=2pt,
        extendedchars=false,
        captionpos=b,
        abovecaptionskip=5pt,
        keepspaces=true,                        
        numbers=left,                                        
        numbersep=5pt,                                    
        showspaces=false,                                
        showstringspaces=false,
        showtabs=false,
        tabsize=2
    }

\definecolor{graytext}{gray}{0.6}

\lstdefinestyle{PostgreSQL}{
    literate={ą}{{\k a}}1
    		 {Ą}{{\k A}}1
             {ż}{{\. z}}1
             {Ż}{{\. Z}}1
             {ź}{{\' z}}1
             {Ź}{{\' Z}}1
             {ć}{{\' c}}1
             {Ć}{{\' C}}1
             {ę}{{\k e}}1
             {Ę}{{\k E}}1
             {ó}{{\' o}}1
             {Ó}{{\' O}}1
             {ń}{{\' n}}1
             {Ń}{{\' N}}1
             {ś}{{\' s}}1
             {Ś}{{\' S}}1
             {ł}{{\l}}1
             {Ł}{{\L}}1,
    keywordstyle=\textbf,
}

\SetAlgorithmName{\LangAlgorithm}{\LangAlgorithmRef}{\LangListOfAlgorithms}
\newcommand{\listofalgorithmes}{\tocfile{\listalgorithmcfname}{loa}}

\renewcommand{\lstlistingname}{\LangListing}
\renewcommand\lstlistlistingname{\LangListOfListings}

\renewcommand{\lstlistoflistings}{\begingroup
\tocfile{\lstlistlistingname}{lol}
\endgroup}

\begin{document}
% ----------Strona tytułowa------------
\title{Eksploracja Danych - Projekt\\
Analiza czynników wpływających na występowanie chorób}
\author{Gabriela Bocheńska, Aleksandra Stachniak, Gabriela Piwar}
\date{\today}
\maketitle

% ----------Spis treści------------
\tableofcontents
\thispagestyle{empty}
\newpage

% ----------Raport------------
\section{Wprowadzenie}

---- WERSJA WIP ---- 

Choroby cywilizacyjne są głównymi przyczynami przedwczesnej śmiertelności i chronicznej niepełnosprawności na całym świecie. Według danych ze Stanów Zjednoczonych [data needed] te schorzenia dotykają miliony Amerykanów, stanowiąc znaczące obciążenie dla systemu opieki zdrowotnej. Do najczęsztyszch możemy zaliczyć cukrzycę, nadciśnienie i udar mózgu.

Cukrzyca to grupa zaburzeń metabolicznych charakteryzujących się wysokim poziomem glukozy we krwi, wynikającym z problemów z produkcją lub działaniem insuliny – hormonu produkowanego przez trzustkę. Istnieją dwa główne typy cukrzycy: typ 1 - wrodzony, gdzie organizm nie produkuje insuliny oraz typ 2 - nabyty, który jest często spowodowany przez niezdrowy tryb życia. Ponieważ
drugi typ jest ściśle związany z nadwagą, brakiem aktywności fizycznej, niezdrową dietą oraz innymi powiązanymi czynnikami, istnieje ogólne ryzyko, że może on dotknąć niemal każdego, kto nie stosuje się do zaleceń zdrowotnych.

Nadciśnienie tętnicze, często nazywane "cichym zabójcą", polega na nieprawidłowo wysokim ciśnieniu krwi w tętnicach, co może prowadzić do uszkodzenia wielu narządów, w tym serca, nerek, mózgu i oczu. Nadciśnienie często rozwija się przez wiele lat bez widocznych objawów, ale może być spowodowane czynnikami takimi jak otyłość, brak aktywności fizycznej, nadmierne spożycie soli, palenie tytoniu oraz czynniki genetyczne. Wczesne wykrycie i leczenie, głównie poprzez zmiany stylu życia i farmakoterapię, są kluczowe dla zapobiegania długoterminowym komplikacjom.

Udar mózgu występuje, gdy przepływ krwi do części mózgu zostaje nagle przerwany, co może być spowodowane zatorem (udar niedokrwienny) lub pęknięciem naczynia krwionośnego (udar krwotoczny). Objawy mogą obejmować nagłe osłabienie, trudności w mówieniu, zrozumieniu mowy, widzeniu, utratę równowagi lub nagłe, silne bóle głowy. Czynniki ryzyka udaru są podobne do tych dla nadciśnienia i cukrzycy, obejmujące niezdrową dietę i brak aktywności fizycznej.

\newpage

\section{Cel projektu}

Celem niniejszego projektu jest zidentyfikowanie populacji osób narażonych na podwyższone ryzyko
wystąpienia przewlekłych chorób takich jak cukrzyca, nadciśnienie tętnicze oraz udar mózgu. Analiza ma na celu wykazanie wpływ stylu życia na rozwój i progresję różnych chorób za pomocą nowoczesnych technik wczesnego wykrywania patologii, pozwalających do szybszej diagnozy wymienionych schorzeń.

Założenia projektu obejmują wnikliwą analizę danych, polegającą na przetworzeniu i interpretacji zawartych informacji w zbiorze. Na jej podstawie w projekcie stworzono kilka modeli predykcyjny, które za zadanie miały identyfikację osób znajdujących się w grupie ryzyka z dużą dokładnością.
        
\section{Zbiór Danych}

\subsection{Opis zbiorów}
Zbiór danych wykorzystany w projekcie pochodzi z witryny Kaggle (Diabetes, Hypertension and Stroke Prediction) i jest wynikiem połączenia trzech różnych zestawów danych dotyczących zdrowia, zawierających odpowiednio wskaźniki cukrzycy, niewydolności serca oraz udaru. Każdy zbiór został wcześniej poddany obróbce w celu wyrównania liczebności klas. W zbiorze znalazły się w nim:

\begin{enumerate}
  \item Pierwszy zestaw danych również pochodzi z Kaggle (Diabetes Health Indicators Dataset). Zawiera 18 cech zdrowotnych związanych z cukrzycą. Do najważniejszych możemy zaliczyć fizyczne cechy takie jak poziom glukozy we krwi, BMI, wiek oraz cechy związane z trybem życia, takie jak ilość spożywanych warzyw i owoców w ciągu dnia, ilość wypijanego alkoholu czy liczba dni kiedy osoba stwierdza u siebie złe samopoczucie. 
  
  \item Drugi zestaw danych, używany w analizie niewydolności serca, pochodzi z witryny smellydatascience.com. Obejmuje on 14 cech,  głównie są to cechy fizyczne pacjenta np. wiek i płeć. Inne zmienne jak ciśnienie krwi, cholesterol oraz wyniki badania EKG są ściśle związane z przeprowadzeniem konkretnych badań. Otrzymane dane ściśle nie opisują trybu życia pacjenta, jednak dają infromację o jego skutkach.
  
  \item Trzeci i ostatni zestaw danych, dostępny także na Kaggle (Stroke Prediction Dataset). Zawiera on podobne wskaźniki zdrowotne jak w pierszym zbiorze, skupiając się na trybie życia pacjenta, takie jak stan cywilny, rodzaj miejsca zamieszkania ale też czy wystąpiły inne choroby u pacjenta związane z układem krążenia. 
\end{enumerate}

\subsection{Oczyszczenie danych}
uzupelnienie brakow, usunięcie wartosci odstajacych, normalizacja/standaryzacja



\subsection{EDA}
wykresy wykresy wykresy

wytypowac profil pacjenta narazonego na chorobe najbardziej
jakie cechy skorelowane? Czy wyszlo cos nieoczekiwanego?


\section{PCA}
\section{Predykcja wystąpienia choroby}
\subsection{Wybór modeli}
\subsection{Ocena jakości}
\section{Analiza czynników ryzyka/Podsumowanie}
udalo sie? 

\begin{thebibliography}{9}
\end{thebibliography}

\end{document}
